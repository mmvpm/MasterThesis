\specialsection{Заключение}

Основным результатом работы является готовый к использованию инструмент, способный на основе инфраструктурных конфигураций микросервисной системы выявлять потенциальные архитектурные нарушения. В процессе достижения этой цели были выполнены следующие задачи.

\begin{itemize}
    \item Проведён обзор предметной области и существующих решений для автоматизированной валидации архитектуры, в ходе которого были выявлены их ключевые недостатки — ограниченная область применения и сложность в интеграции с корпоративной инфраструктурой.
    \item Разработан инструмент, предоставляющий гибкие возможности для описания архитектурных правил любой сложности и их автоматической проверки применимо к конкретной архитектуре.
    \item Реализован простой способ написания новых правил для шаблонных ситуаций через YAML-конфигурации, что ускоряет процесс добавления стандартных проверок.
    \item В инструмент заложен набор правил по умолчанию, проверяющих соблюдение распространённых архитектурных антипаттернов.
    \item Обеспечена интеграция решения с используемыми в Яндекс.Вертикалях системами и инструментами разработки, включая возможность сравнения архитектурных изменений в ветках и получение уведомлений через Telegram.
    \item Измерена эффективность инструмента на четырёх реальных микросервисных системах. Анализ полученных результатов показал способность инструмента выявлять десятки потенциальных архитектурных нарушений, формируя числовую метрику архитектурного технического долга анализируемого проекта.
\end{itemize}
